\chapter{Decisiones estratégicas de marketing}

\section{Selección del segmento y posicionamiento}
\subsection{Criterios de segmentación}
Proponemos los siguientes criterios de segmentación:
\begin{itemize}
    \item \textbf{Nivel Socio-Económico}, tenemos 3 clases: A, B, C
    \item \textbf{Edad}, jóvenes (12-25), adultos (26-65), mayores (65 o más)
    \item \textbf{Tipo de persona}, particulares o empresas
    % distritos: central, periferico
\end{itemize}

\subsection{Target}
Nuestro target son jóvenes y adultos de clases socio-económicas A y B. Los jóvenes y adultos representan el 66\% de nuestra población y las clases económicas A y B representan el 62\% de peruanos.

Creemos que el segmento que nos puede generar la mayor demanda y por lo tanto, en el que debemos enfocarnos más es el segmento de los jóvenes de clase B.
% tamanio de segmentos
% demanda actual de segmentos

\subsection{Posicionamiento esperado en el mercado meta}
Por el momento no hay ninguna barrera de entrada, puesto que no hay una empresa posicionada en el rubro de courier en nuestro país. Nuestra ventaja será establecernos como una empresa seria en la que los clientes confían y que sea ampliamente conocida. Esperamos que las personas relacionen el nombre de nuestra empresa con la palabra courier, sean o no clientes.
% Ventaja competitiva que nos permita posicionarnos
% Nuestro nombre sea sinonimo de courier