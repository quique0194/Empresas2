\chapter{Decisiones operativas de marketing}

\section{Productos}

\section{Precios}

\subsection{Identificación de costos}

Identificamos los siguientes costos fijos:

\begin{itemize}
\item \textbf{S/. 256}: Servidores en la nube, Amazon EC2, máquinas del tipo t2.large (2 CPUs, 8GB RAM)
\item \textbf{S/. 600}: Alquiler de la oficina de desarrolladores
\item \textbf{S/. 150}: Recibos de luz y agua
\item \textbf{S/. 120}: Internet Claro de 8Mb
\item \textbf{3 x S/. 2000}: Salarios
\item \textbf{S/. 300}: Publicidad en redes sociales
\end{itemize}

No identificamos ningún costo variable.

\subsection{Política de precios}

Queremos brindar un servicio de calidad y fácil de usar. Pretendemos una penetración rápida y profunda en el mercado reduciendo el margen de ganancia al mínimo en un inicio, y luego incrementándolo una vez que los usuarios se acostumbren a la aplicación.

Tenemos dos frentes donde cobrar dinero. En primer lugar tenemos a los usuarios finales, a quienes se les ofrece un servicio de calidad por un precio razonable, no esperamos obtener ganancias de este grupo. De donde esperamos obtener ganancias es de cobrarle una comisión a los taxistas por cada encomienda que se les consiga. Durante los dos primeros meses que el taxista utilice la aplicación, gozará del 100\% de los beneficios de su trabajo, luego se le cobrará una comisión del 10\%. Esperamos que los taxistas ya familiarizados con la aplicación y conscientes de su beneficio, luego de dos meses de prueba gratis, estén dispuestos a pagar la comisión.

\section{Promoción}

\section{Distribución}

\section{Personas}

\section{Evidencia física}

\section{Procesos}