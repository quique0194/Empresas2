\chapter{Diagnóstico situacional}

\section{Análisis externo}

\subsection{Análisis del entorno}
\subsubsection{Análisis Político}
Actualmente en el Perú se tiene una legislación de libre mercado. A pesar de que nos encontramos en pleno proceso electoral, las propuestas de los dos candidatos restantes no alteraran este hecho. Debido a que nuestro emprendimiento es digital, no hay una entidad reguladora, más allá de los impuestos que hay que tributar. Actualmente se está poniendo énfasis en el financiamiento de proyectos tecnológicos, por lo que será relativamente sencillo levantar capital para el proyecto.
% Politicas de transporte, de seguridad en los taxis

\subsubsection{Análisis Económico}
La situación económica del país ha mejorado en el presente año. Prueba de esto es que el PBI se ha incrementado en un 4.4\% en el primer trimestre del 2016. Más aún, en el área de servicios, se ha incrementado un 4.8\%. Por otro lado, el año pasado se aprobó una ley que incentiva a las empresas a invertir en tecnología, haciendo que sea posible deducir los gastos en tecnología hasta en un 175\% para el cálculo del IR.
% Analisis economico de los servicios de courier y tecnologia

\subsubsection{Análisis Social}
La aceptación social de la tecnología también ha mejorado. Siendo que el número de usuarios de smartphones en el país se ha incrementado, sin embargo, esto sucede sólo en los centros urbanos. En los pueblos más alejados todavía no se asienta el uso de esta tecnología. El uso de dispositivos móviles es transversal a las clases sociales y a los estilos de vida. Finalmente, es importante notar que los consumidores prefieren descargar apps gratuitas.
% costumbres de nuestra sociedad que pueden beneficiar o afectar la propuesta
% servicios de delivery poco usados
% desconfianza de la gente

\subsubsection{Análisis Tecnológico}
Existe toda la tecnología necesaria para desarrollar este proyecto. También hay una tecnología similar (Easy taxi), pero orientada al transporte de pasajeros, sin contar con un servicio de courier. El inconveniente es que esta tecnología estará restringida a usuarios de smartphones con conexión a internet. La tecnología sustituta que se usa normalmente para transporte de pasajeros es una llamada telefónica a las empresas de taxi. Sin embargo, la entrega de paquetes no tiene una tecnología sustituta, ya que la única forma de enviar un paquete a otro punto de la ciudad es llevarlo tu mismo o crear tu propio sistema de entregas en caso que seas una empresa.
% la gente tiene smartphones que son nuestra plataforma tecnologica, taxistas con gps
% que usaremos: servicios de amazon ec2, neo4j, google maps


\subsubsection{Análisis Ambiental}
% No hay, quiza el humo de los carros, pero nos vale verga
% Podriamos optimizar el transporte y por lo tanto disminuir las emisiones de CO2


\subsection{Análisis general de la industria}
\subsubsection{Tamaño del mercado}
El 80\% de la población entre 12 y 70 años ya tenían un smartphone en Abril del año pasado. De este grupo, el 52\% utiliza internet en sus equipos. Teniendo en cuenta la población de Arequipa en el último censo del 2012 y la tasa de crecimiento de la población, se estima que la población actual de Arequipa es de 1325000 habitantes. De estos, el 75\% vive en la capital (que es nuestro objetivo). Por lo tanto el tamaño del mercado total sería aproximadamente de unos 413400 individuos. Si logramos alcanzar al 10\% durante el primer año, tendríamos un total de 41340 clientes.
% gráficos
% aterrizarlo a la gente que usa servicio de courier

\subsubsection{Estacionalidad}
No existe estacionalidad en nuestro servicio, ya que se utiliza constantemente a lo largo del año.

\subsubsection{Tendencias del sector}
Actualmente la tendencia más difundida es salir a la calle y parar un taxi cualquiera, y en situaciones donde la seguridad es más importante, se llama a una empresa de confianza para que envíe un taxi al lugar acordado. Para la entrega de paquetes la tendencia es designar un lugar para que la otra persona lo recoja o llevarlo uno mismo. No existe un sistema de entregas interurbano. Aunque si existen sistemas de entregas a otros departamentos como Olva Courier, estos no realizan entregas entre 2 puntos distintos de la ciudad, el paquete que ellos entregan debe ser dejado y recogido en sus oficinas en 2 ciudades distintas.
% llevar el paquete
% recoger el paquete
% enviar a alguien de confianza con el paquete
% empresas: su propia forma de delivery


\subsubsection{Crecimiento potencial}
Contamos con dos formas de crecimiento potencial, la primera es captar más clientes en la ciudad de Arequipa y pasar de un 10\% que es la meta para el primer año a un 50\% en el año siguiente. Y la otra forma de crecimiento es expandirnos a más ciudades del país, siendo las más llamativas las ciudades de Lima y Trujillo.

\subsubsection{Infraestructura}
No se necesita infraestructura física, ya que se utilizará a los propios taxistas de la ciudad. Sólo hacen falta 2 aplicaciones móviles, una para los usuarios, que se distribuirá libremente, y otra para los taxistas, que se les otorgará luego de pasar por un proceso de verificación de identidad.
% Infraestructura computacional
% local donde se reciban reclamos, se verifiquen los taxistas
% oficina de trabajo

\subsubsection{Precios del sector}
Los precios para las carreras de los taxis varían desde los 5 soles hasta los 20 aproximadamente. La idea es que para el transporte de paquetes los precios sean menores que el transporte de personas, ya que en un mismo taxi se pueden llevar varios paquetes o incluso un cliente.
% precio de los taxis
% precio del mandadito
% precio de olva courier
% precio de delivery
% precio de taxitel courier

\subsubsection{Márgenes de la industria}
Se espera cobrar un 10\% del precio total pagado por los clientes. El otro 90\% se quedará con el taxista para su propio beneficio y mantención de su vehículo, así como el gasto de la gasolina.
% cuanto ganan los taxistas
% cuanto ganan olva courier, mandadito, taxitel, etc


\subsection{Análisis de las fuerzas competitivas de la industria}

\subsubsection{Proveedores y su poder de negociación}
Los proveedores son las personas que impulsaran el negocio, en este caso nosotros mismos y taxistas varios, como también inversionistas de todo tipo que quieran invertir en el producto, para ampliar los parámetros de la empresa y llegar a la visión predispuesta anteriormente.
% ver la cantidad de proveedores

\subsubsection{Rivalidad entre las empresas que compiten en el mercado}
La rivalidad se da por medio de la competencia, en este caso hay empresas pequeñas que ofrecen este servicio, pero no están avanzadas, o solo funcionan para pequeñas locaciones. Existe una empresa que es EASYTAXY que es lo mas acercado a nuestro producto pero, no brinda envió de paquetes, por lo que seria un punto a favor.
% cuanta competencia hay, como captan clientes la competencia

\subsubsection{Nuevos competidores y sus posibilidades de entrada}
Con lo referente a la posibilidad de nuevos competidores esta abierta, ya que cualquiera puede acceder a este servicio y hacer una copia de este servicio, o incluso mejorarla, pero lo que les sera difícil es conseguir los clientes, y su confianza, por lo que muchos tenderán a cerrar.
% barreras de entrada

\subsubsection{Desarrollo de productos sustitutos}
Frente a la posibilidad de fracaso del servicio brindado, se ve que  actualmente no se cuenta con un servicio serio de rutas en tiempo real de un servicio de transporte, por lo que bastaría con crear una aplicación, en donde  registre la placa del  vehículo en donde voy y otra persona podría ver donde estoy realmente.
% que otros productos se pueden utilizar en lugar del nuestro
% taxis, transporte publico, courier dentro de la empresa

\subsubsection{Compradores y su poder de negociación}
En este campo se da dos tipos de servicio:
\begin{itemize}
    \item Envío de un paquete,las personas pagan por el servicio de taxi incluyendo la comisión de la empresa, por el servicio.
    \item Traslado de una persona, al igual que el anterior, la persona paga por el taxi y el servicio de la empresa, y todo esto manejado por tarjetas o por efectivo. 
\end{itemize}
% cuantos compradores hay

\section{Análisis interno}

\subsection{Análisis AMOFHIT}
\subsubsection{Administración y Gerencia}
Los dueños de la empresa son personas sociables, amigables, y sobre todo exigentes con su trabajo, y en el posible caso de un incidente, todo estará completamente organizado, ya que se cuenta con una serie de conocimientos, que hagan posible, que todo funcione como debería ser. 
% cuatro fundadores: 2 computer science
% un webon de marketing
% un administrador pingon

\subsubsection{Marketing y Ventas}
El producto esta elaborado para personas entre 18 y 50 años aproximadamente, el servicio se hace lo mas cómodo posible para el cliente.
En cuanto al precio varia, dependiendo de la distancia que se va a recorrer, y el nivel de seguridad que se necesita para trasladar ese producto.
Con respecto a la plaza, se tomaría a partir del software ya existente como es el caso de EASYTAXI. Siempre habrá promociones, para poder entrar a las personas, y es que si no se construye un buen sistema, no se genera competitividad, solo así se ganara al cliente, y se hará que nuestro servicio sea de calidad.  
% publicidad
% promociones
% forma de ganar dinero: comisiones
\subsubsection{Operaciones Logísticas}
El costo del producto, viene dado por las horas hombre que se programo, la luz que se gasto,el celular que se compro para hacer las pruebas necesarias, la gasolina que se invirtió para movilizarse. Es por eso que el producto se vende dependiendo de que tan largo sea el recorrido, haciendo una valoración entre la gasolina, las horas hombre,mantenimiento del carro,a eso se le suma la seguridad, si el paquete es muy delicado, o muy valioso, implica un gasto adicional, ya que se pagaría a mas personas para que cuiden el paquete, y a eso se le adiciona los gasto indirectos como es el pago a las personas que trabajan en finanzas, en sistemas, en recursos humanos.
% Como validar choferes
% algoritmos
% como funcionara el sistema gps de seguridad

\subsubsection{Finanzas y Contabilidad}
Al principio el trabajo estará dedicado a dos personas, quienes programaran el software, por lo que no habrá mucho problema, pero una vez que ya se haya programado  y ya la empresa este moviéndose se necesitara contratar un contador que declare a la sunat, por nuestros ingresos.
% como nos vamos a financiar
% capital de fundadores
% recaudar capital semilla de inversores y concursos de startups

\subsubsection{Recursos Humanos}
No habrá preferencia por ningún trabajador, se le pagara lo que corresponde a cada uno, de acuerdo al labro desempeñado.Se tiene que conocer a cada trabajador, para fomentar un ambiente de familia, en donde cada uno se sienta parte de un todo. Los trabajadores tendrán incentivos, por una cierta cantidad de paquetes enviados con éxito, y los días importantes se celebraran en familia, Se organizara un día especial en el mes que sera de relajo y amenización entre todos. 
% como vamos a conseguir trabajadores: practicantes, bajo salario, medio tiempo
% que perfil buscamos: tercio superior, con ganas de aprender

\subsubsection{Sistema de información y Comunicaciones}
Se llevara un control total desde que se hizo el contrato para llevar un producto, hasta que llega a su destino esto mediante un GPS o un sistema radial. Esto para un mejor control ante perdidas. Además con estos datos se hará un cuadro estadístico, para tener información de que tan bueno es nuestro servicio y mejorar en lo que sea necesario.
% no se de q xuxa trata, investiga puto
% taxis contaran con internet y GPS
% las comunicaciones seran por medio de la aplicacion

\subsubsection{Tecnología Investigación y desarrollo}
Cuando comience el negocio, se tendrá un servidor que estará valorizado en 200 dolares, ya que tendrá gran capacidad, y a medida que pase el tiempo, se obtendrá mas servidores si es que es necesario, debido a la demanda del uso del software. Además se investigara la posibilidad de ampliar el negocio a otros sectores del Perú o de latino América, y la posibilidad de que el cliente pueda ver donde se encuentra su paquete en tiempo real.  
% como vamos a innovar
% trabajos futuros: transporte de pasajeros, compartir taxi, integración con asistentes de voz (siri, cortana)

\subsection{Recursos y capacidades de la empresa}
Si bien hay varios tipos de recursos, a nivel físico, se tiene un ordenador pequeño, un par de celulares, dos laptops en donde se codificara el software, a nivel de recurso humano se tiene a dos personas por el momento, quienes codificaran el software, se espera que mas personas con movilidad se unan a esta causa, y a nivel organizativo por el momento se cuenta con dos personas, pero a medida que avance el tiempo y la marca sea reconocida se empezara a contratar mas personal administrativo que sean de ayuda a la organización.
%tablita maricona, investiga puto gay

\subsection{Ventaja competitiva actual}
En el mercado, la mayor parte de personas usa un smartphone, por lo que se le es mas fácil pedir un servicio, pero esto es un problema cuando se trata de enviar un paquete, ya que la seguridad es un factor muy importante. Además existen varios software que hacen este tipo de cosas, pero no son muy reconocidas, lo que se busca es que este software ayude a las personas a tener un envió seguro de sus paquetes, y eso es lo generara una ventaja competitiva. 
% seguridad
% barato
% facil de usar: tanto como para clientes como para transportista

\subsection{Estructura organizativa}
Actualmente se cuenta con una estructura lineal, ya que el gerente es el dueño, y además es una pequeña empresa que, recién esta brotando inmediatamente después se contaría con una estructura mono funcional en donde solo se tenga un objetivo,enviar paquetes. Lo que se busca es una estructura mucho mas especifica que controle la gestión de los envíos a detalle o incluso mejorar el servicio, pero solo se lograra a medida que avance el tiempo y se vena resultados.
% Cabeza 2 CEOs
% areas de: marketing, desarrollo, relaciones transportistas