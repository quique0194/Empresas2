\chapter{Diagnóstico situacional}

\section{Análisis externo}

\subsection{Análisis del entorno}
\subsubsection{Análisis Político}
Actualmente en el Perú se tiene una legislación de libre mercado. A pesar de que nos encontramos en pleno proceso electoral, las propuestas de los dos candidatos restantes no alteraran este hecho. Debido a que nuestro emprendimiento es digital, no hay una entidad reguladora, más allá de los impuestos que hay que tributar. Actualmente se está poniendo énfasis en el financiamiento de proyectos tecnológicos, por lo que será relativamente sencillo levantar capital para el proyecto.

\subsubsection{Análisis Económico}
La situación económica del país ha mejorado en el presente año. Prueba de esto es que el PBI se ha incrementado en un 4.4\% en el primer trimestre del 2016. Más aún, en el área de servicios, se ha incrementado un 4.8\%. Por otro lado, el año pasado se aprobó una ley que incentiva a las empresas a invertir en tecnología, haciendo que sea posible deducir los gastos en tecnología hasta en un 175\% para el cálculo del IR.

\subsubsection{Análisis Social}
La aceptación social de la tecnología también ha mejorado. Siendo que el número de usuarios de smartphones en el país se ha incrementado, sin embargo, esto sucede sólo en los centros urbanos. En los pueblos más alejados todavía no se asienta el uso de esta tecnología. El uso de dispositivos móviles es transversal a las clases sociales y a los estilos de vida. Finalmente, es importante notar que los consumidores prefieren descargar apps gratuitas.

\subsubsection{Análisis Tecnológico}
Existe toda la tecnología necesaria para desarrollar este proyecto. También hay una tecnología similar (Easy taxi), pero orientada al transporte de pasajeros, sin contar con un servicio de courier. El inconveniente es que esta tecnologia estará restringida a usuarios de smartphones con conexión a internet. La tecnología sustituta que se usa normalmente para transporte de pasajeros es una llamada telefónica a las empresas de taxi. Sin embargo, la entrega de paquetes no tiene una tecnología sustituta, ya que la única forma de enviar un paquete a otro punto de la ciudad es llevarlo tu mismo o crear tu propio sistema de entregas en caso que seas una empresa.


\subsection{Análisis general de la industria}
\subsection{Análisis de las fuerzas competitivas de la industria}

\section{Análisis interno}

\subsection{Análisis AMOFHIT}
\subsection{Recursos y capacidades de la empresa}
\subsection{Ventaja competitiva actual}
\subsection{Estructura organizativa}