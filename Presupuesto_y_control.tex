\chapter{Presupuesto y control}

\section{Presupuesto y retorno de la inversión}

\begin{figure}[htb]
\centering
\includegraphics[width=1\textwidth]{./img/van_tir}
\caption{Cálculo del VAN y el TIR} \label{fig:van_tir}
\end{figure}

\begin{figure}[htb]
\centering
\includegraphics[width=0.7\textwidth]{./img/taxis}
\caption{Crecimiento del número de clientes taxistas} \label{fig:taxis}
\end{figure}

Para calcular el VAN (Valor Actual Neto) y el TIR (Tasa Interna de Retorno), asumimos la siguiente información. La duración del proyecto será de 4 años. La suma de nuestros costos fijos (ver sección \ref{sec:costos}) es de S/217,512.00 los dos primeros años, luego aumenta a S/277,512.00 debido a la inversión en publicidad. Tomamos como inversión inicial S/180,000.00 que es el costo necesario para contratar 5 programadores más, ya que durante el primer año del proyecto se necesitará mucho esfuerzo en el desarrollo de la aplicación. No hace falta comprar equipo especial al inicio del proyecto, pues se asume que los trabajadores tienen sus propias herramientas de desarrollo. Respecto a los gastos variables, sólo tenemos las calcolmanías para los taxis, que calculamos multiplicando el número de taxis clientes por S/10.00; es importante notar que cada año se le renuevan las calcomanías a nuestros clientes taxistas.

Para calcular los ingresos, partimos del dato de que hay más de 30 mil taxis en Arequipa. Para efectos prácticos nosotros redondeamos esta cifra a 30 mil exactos. Nuestra proyección (Figura \ref{fig:taxis}) de crecimiento nos dice que durante el primer año, un 2\% de taxistas utilizarán nuestra aplicación, para el segundo año serán un 6\%, y para el tercer y cuarto año serán un 10\% y 15\% respectivamente. Luego, el número promedio de carreras diarias de un taxista será de 1, más adelante veremos como se justifica esto. Si cobramos una comisión equivalente a S/. 0.50 por cada carrera, entonces nuestra ganancia promedio al día por taxista que usa nuestra aplicació es de S/0.50. Tomaremos como ejemplo el primer año, donde deberíamos tener el 2\% de taxistas trabajando con nosotros, eso es un total de 600 taxistas. Luego, las ganancias por taxista son de S/0.50 diarios que multiplicado por los 600 taxistas y los 365 días del año nos da un bonito S/109,500.00 que son los ingresos del primer año. Los ingresos de los años siguientes se calculan de forma análoga.

Para justificar el hecho de que cada taxista hará en promedio una carrera diaria, nos basamos en el número total de PYMES en la ciudad de Arequipa. Sabiendo que hay aproximadamente 66 000 PYMES en Arequipa, podemos especular que al día un 10\% de ellas harán por lo menos un envío. Obviamente estamos hablando de un momento cuando la idea de enviar paquetes por medio de EasyCourier está más o menos difundida, gracias a nuestras inversiones en publicidad. Al cuarto año tendremos 4500 taxistas para cubrir 6600 envíos diarios, de ahí obtenemos que cada taxista hará en promedio un envío diario.

Asumimos una tasa impositiva de 30\% y una tasa de descuento de 20\% para calcular el VAN. El resto sólo son fórmulas. Finalmente obtenemos un TIR de 34.4\%.

\section{Cronograma de implementación}

\begin{figure}[htb]
\centering
\includegraphics[width=1\textwidth]{./img/cronograma}
\caption{Cronograma de implementación} \label{fig:cronograma}
\end{figure}

El cronograma se encuentra bosquejado en la figura \ref{fig:cronograma}. El plan a grandes rasgos es crear rápidamente un MVP para poder levantar inversiones y lanzar el proyecto. Buscar al resto de las personas del equipo, que por ahora sólo somos dos fundadores y centrarnos durante dos meses en las dos partes más críticas del sistema: la verificación de la identidad de los taxistas y el servicio de courier propiamente dicho. Luego de esto se inicia una campaña de marketing agresiva y se trata de hacer convenios con las empresas de taxi más reconocidas. A la par, se mantiene una cultura de constante análisis y mejora dentro de la empresa, de forma que nos podamos mantener innovando.


\section{Mecanismos de control}

Para controlar nuestros objetivos financieros, nos reuniremos una vez al mes a discutir el impacto de nuestras acciones del último mes y decidir si conservamos la misma política o cambiamos hacia alguna nueva propuesta.

Para controlar las operaciones del proyecto, tenemos bastantes métodos, pues queremos ofrecer un servicio seguro, entre estos métodos tenemos la verificación de la identidad de los taxistas para que puedan entrar al sistema, el seguimiento por GPS que se le hace a los paquetes y la propia calificación que colocan los usuarios.

En cuanto al seguimiento que le haremos a nuestros trabajadores, consistirá básicamente en seguir la metodología scrum de reunirnos una vez al día todo el equipo para comentar de forma rápida que hemos hecho el día anterior, que planeamos para el día siguiente y si hemos tenido algún problema en ese lapso de tiempo.

Finalmente para controlar el nivel de satisfacción del cliente, tomamos estadísticas y métricas a partir del uso de la propia aplicación, puesto que sería complicado pedirle a nuestros clientes que nos llenen una encuesta.